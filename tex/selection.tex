\section{Selection on Kin Cooperating Group Structure}

\begin{frame}
\begin{itemize}
\item expect transition to occur
\item scalable
\item no simulation of ``realistic'' physics, abstract instead
\end{itemize}
\end{frame}

\begin{frame}{Resource Collection}
\begin{itemize}
\item need to anticipate
\item need to activate
\item communication with neighbors
\end{itemize}
\end{frame}

\begin{frame}{Channels, Signals, \& Resource}
\begin{itemize}
\item register into explicit cooperating groups called channels
\end{itemize}
\end{frame}

\begin{frame}{Channels, Signals, \& Resource}

\begin{figure}

\begin{columns}
\begin{column}{0.45\textwidth}

  \foreach \n in {1,...,6}{%
  \includegraphics<\n>[width=\textwidth]{explanatory_sep/r-\n.pdf}%
  }%

\end{column}
\begin{column}{0.45\textwidth}

  \foreach \n in {1,...,6}{%
  \includegraphics<\n>[width=\textwidth]{explanatory_sep/s-\n.pdf}%
  }%

\end{column}
\begin{column}{0.1\textwidth}

  \foreach \n in {1,...,6}{%
  \includegraphics<\n>[width=\textwidth]{explanatory_sep/rs-\n.pdf}%
  }%

\end{column}
\end{columns}

\begin{columns}
\begin{column}{0.45\textwidth}
  \begin{subfigure}[b]{\textwidth}
  \caption{Resource}
  \end{subfigure}
\end{column}

\begin{column}{0.45\textwidth}
  \begin{subfigure}[b]{\textwidth}
  \caption{Channel map}
  \end{subfigure}
\end{column}

\begin{column}{0.1\textwidth}
  \begin{subfigure}[b]{\textwidth}
  \end{subfigure}
\end{column}

\end{columns}

\caption{
Frame-by-frame animation of a small same-channel network, activation signal, and resource wave interaction and effect on resource collection.
}

\end{figure}

\end{frame}

\begin{frame}{Channels, Signals, \& Resource}

\begin{figure}

\begin{columns}
\begin{column}{0.45\textwidth}

  \foreach \n in {1,...,8}{%
  \includegraphics<\n>[width=\textwidth]{explanatory_sep/r-\n.pdf}%
  }%

\end{column}
\begin{column}{0.45\textwidth}

  \foreach \n in {1,...,8}{%
  \includegraphics<\n>[width=\textwidth]{explanatory_sep/m-\n.pdf}%
  }%

\end{column}
\begin{column}{0.1\textwidth}

  \foreach \n in {1,...,8}{%
  \includegraphics<\n>[width=\textwidth]{explanatory_sep/rm-\n.pdf}%
  }%

\end{column}
\end{columns}

\begin{columns}
\begin{column}{0.45\textwidth}
  \begin{subfigure}[b]{\textwidth}
  \caption{Resource}
  \end{subfigure}
\end{column}

\begin{column}{0.45\textwidth}
  \begin{subfigure}[b]{\textwidth}
  \caption{Channel map}
  \end{subfigure}
\end{column}

\begin{column}{0.1\textwidth}
  \begin{subfigure}[b]{\textwidth}
  \end{subfigure}
\end{column}

\end{columns}

\caption{
Frame-by-frame animation of a medium-sized same-channel network, activation signal, and resource wave interaction and effect on resource collection.
}

\end{figure}

\end{frame}

\begin{frame}{Channels, Signals, \& Resource}

\begin{figure}

\begin{columns}
\begin{column}{0.45\textwidth}

  \foreach \n in {1,...,9}{%
  \includegraphics<\n>[width=\textwidth]{explanatory_sep/r-\n.pdf}%
  }%

\end{column}
\begin{column}{0.45\textwidth}

  \foreach \n in {1,...,9}{%
  \includegraphics<\n>[width=\textwidth]{explanatory_sep/l-\n.pdf}%
  }%

\end{column}
\begin{column}{0.1\textwidth}

  \foreach \n in {1,...,9}{%
  \includegraphics<\n>[width=\textwidth]{explanatory_sep/rl-\n.pdf}%
  }%

\end{column}
\end{columns}

\begin{columns}
\begin{column}{0.45\textwidth}
  \begin{subfigure}[b]{\textwidth}
  \caption{Resource}
  \end{subfigure}
\end{column}

\begin{column}{0.45\textwidth}
  \begin{subfigure}[b]{\textwidth}
  \caption{Channel map}
  \end{subfigure}
\end{column}

\begin{column}{0.1\textwidth}
  \begin{subfigure}[b]{\textwidth}
  \end{subfigure}
\end{column}

\end{columns}

\caption{
Frame-by-frame animation of a large same-channel network, activation signal, and resource wave interaction and effect on resource collection.
}

\end{figure}

\end{frame}

\begin{frame}{Signals \& Resource}
  \vspace{4ex}
  \input{fig/signals_resource.tex}
\end{frame}

\begin{frame}{Channel Inheritance}
  \vspace{8ex}
  \input{fig/channel_inheritance.tex}
\end{frame}

\begin{frame}{Excluded Channel Dynamics}
  \vspace{6.6ex}
  \begin{figure}
\begin{columns}
  \begin{column}{0.5\textwidth}
    \onslide<2->{\colorbox{extralightgray}{\hspace{0.136\textwidth}\includegraphics[width=0.728\textwidth,trim= 0 -83 0 -83]{plastic_channel}\hspace{0.136\textwidth}}}
  \end{column}%
  \begin{column}{0.5\textwidth}%
    \onslide<3->{\colorbox{extralightgray}{\includegraphics[width=\textwidth]{channel_collision_offspring}}}
  \end{column}
\end{columns}%
\begin{columns}
  \begin{column}{0.5\textwidth}
    \onslide<2->{
    \begin{subfigure}[b]{\textwidth}
    \caption{during-lifetime channel change}
    \end{subfigure}
    }
  \end{column}
  \begin{column}{0.5\textwidth}
    \onslide<3->{
    \begin{subfigure}[b]{\textwidth}
    \caption{non-inherited channel match\footnotemark}
    \end{subfigure}
    }
  \end{column}
\end{columns}
\caption{Illustration of excluded channel dynamics.}
\end{figure}
\only<3>{\vspace{-1.642ex}\footnotetext{infrequent, non-inducible}}%

\end{frame}

\begin{frame}{Intuition}
key idea: selection on regulation of same-channel group size \& shape
\begin{itemize}
  \item too small: low resource-collection rate
  \item too big: net negative resource collection due to erroneous activation
\end{itemize}

more generally speaking,

Selection on Kin Cooperating Group Structure

\end{frame}
